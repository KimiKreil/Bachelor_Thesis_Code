\section{Discussion}

There are, however, some limitations to aperture photometry and the residual map it produces. All sources in the image are modelled individually with a point source model, where the flux is a free parameter that rescales the profile to fit the source. The flux used is the total flux within a fixed aperture, hence the aperture size is an assumption we enforce on all sources, although an adaptive aperture may have been more appropriate. This is not always a problem as long as the aperture size is relatively large, including all flux intrinsic to the source, and the surrounding is background which values are centred around zero. The problem arises when there is a  blended pair of galaxies, and the intrinsic flux cannot be properly estimated or divided between the sources. To address the drawbacks of aperture photometry, we explore a different approach of creating a residual image.

\begin{itemize}
    \item What are pros and cons of the Farmer method? Is the Farmer residual image better than Iary's old one?
    \item Is ML really better than visual classification? pros and cons of t-SNE.
    \begin{itemize}
        \item They work together since we do need some tracers to be able to classify anything.
        \item PRO: ML is significantly faster. Reduces the sample to look at.
        \item CON: not a direct pipeline, the t-SNE embedding has to be reconfigured everytime we add new data. Does run quite fast so not an imediate problem.
        \item PRO: We used t-SNE on not the most optimal data. Perhaps it will perform better with a better and more robust residual map than can be developed from the farmer catalogue.
        \item PRO: if we design the labeling of tracers for this purpose, i.e. also labeling bad candidates it might perform a little better. (not sure about this one but worth looking into)
        \item PRO: we can chose the sensitivity. Do we want a larger sample with more contamination or a pure smaller sample. Easy to make that choice in ML, just changing fmin, and we can chose which result we want to use depending on the application.
    \end{itemize}
    \item How the COSMOS dropout galaxies look like? (eg colors, etc) Do they "agree" with previously discovered droupouts? Compare to results from \cite{Wang_2019} and \cite{Alcalde_Pampliega_2019}
    \item A very quick discussion/speculation about how these objects fit into the "big picture" of galaxy evolution.
\end{itemize}


