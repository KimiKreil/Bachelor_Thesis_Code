Recent work revealed a significant population of "hidden galaxies" in the early universe (redshift $z>3$), which systematically escaped optical telescope imaging due to high dust content. These peculiar "H-dropout galaxies" can now be identified in state-of-the-art infrared surveys, where the impact of dust is less strong.
This project proposes a semi-automated method for the identication of the most reliable candidates of "H-dropout galaxies". The technique takes advantage of unsupervised and semi-supervised machine learning to classify galaxies. In particular, we use t-distributed Stochastic Neighbor Embedding (t-SNE) for dimensionality reduction on unlabeled data, consisting of cutouts of telescope images in various bands. With information from the COSMOS catalogue we construct a combined model of the telescope image. Subtracting the model from the the scientific image we produce a residual map that is used for the detection of the objects constituting our sample. Through visual inspection of the residual image, a sample of respectively promising and unfit candidates is labelled to allow for semi-supervised classification of galaxies according to the vote of k neighbours. With this method we are able to classify $\sim300$ robust candidates from the entire sample of $\sim3000$ galaxies, thus reducing the sample to be visually insepcted by a factor of 10. This method will allow identification and classification of distant, optically faint galaxies among the millions of object that will be detected in next-generation surveys such as \textit{Euclid}. Upon confirmation of the candidates, this might lead to a challenge of our understanding of galaxy evolution and/or provide insight into the physical features of intrinsically faint, distant galaxies.

\textcolor{red}{Include link to github with code}
\textcolor{blue}{Upon confirmation of the candidates, through spectral analysis, this can lead to interesting further research. If the galaxies are intrinsically faint, and thus quiescent at a high redshift, they can challenge our understanding of galaxy evolution. If the galaxies are not intrinsically faint, they might prove to contain unprecedented amounts of dust and will complement our understanding of the physical features of faint galaxies.}
