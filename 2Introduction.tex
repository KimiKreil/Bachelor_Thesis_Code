\section{Introduction}
Here the introduction should go. This should be split into three smaller sections:
\begin{itemize}
    \item (1): General introduction to galaxy evolution (our big question).
    \item (2): Specific motivation of this thesis (our subproblems - the way we chose to answer the big question, and specific motivations from other articles)
    \item (3): Anticipations (what we will try to do)
    modelling the detected object and subtract them to make a residual image; 
    identify the residual objects that have an astrophysical interest; 
    [analyse these interesting objects]. 
\end{itemize}
Finally include and overview of what can be expected in each section.
\cite{Steinhardt_2020}
\cite{Wang_2019}
\cite{Alcalde_Pampliega_2019} \\


\textbf{QUICK DRAFT}: \\
Galaxies consist of stars, gas and dust. A galaxy can thus "die" when all of its stars has burned out, and it has used all of its star-forming gas. Our understanding of galaxy evolution comes from *sources*, and it is believe that galaxies can form and die in a time span of x billion years. Papers found that massive red galaxies, old, quiecent galaxies about to die or dead can be found at redshift z>3, which indicate that they can die within 2 billion years. This challenges our understanding of galaxy evolution, and is what this project sets out to investigate. So far these galaxies have been found manually though a cumbersome proces, which only yields a few robust candidades, not enough to challenge our understanding entriely. This paper sets out to find a quicker way to do this by using unsupervised machinelearning to limit candidades with high probability of these old H-dropout galaxies, this can hopefully allow for more effective detection of this newly found galaxy type, and a bigger quantity that later with the james webb telescope perhaps can get more quality data, will allow us to expand our understanding of galaxy evolution.

How will this be done: modelling objects, create residual image, manifold learning, automatic candidate detection ready for in depth analysis afterwards. anticipations: this will not automatically find the objects we are interested but instead isolate quickly the better candidates which automates the proces a little. Not a finished pipeline, but a tool for detecting these objects faster.
