\section{Introduction}
Reaching a complete understanding of the formation and evolution of galaxies is one of the central challenges of modern astronomy. To be able to study the early stages of galaxy evolution, we look for progenitors of present-day galaxies. In an astrophysical context, studying the early universe implies looking for distant galaxies with a high redshift, i.e. galaxies which emitted the light we observe at an earlier point in cosmic time. For example, a galaxy at redshift $z=5$ (corresponding to a luminosity distance of $3\,\si{Gpc}$) emitted light $\sim1$ billion years after the Big Bang
\cite{2006PASP..118.1711W_calculator}. \\

One method of identifying a distant galaxy is observing the spectrum and directly measure the redshift, using the cosmological redshift formula in eq. \ref{redshift}:
\begin{equation}
    \lambda_\mathrm{obs} = (1+z)  \lambda_\mathrm{emit}
    \label{redshift}
\end{equation}
where $\lambda_{obs}$ is the observed wavelength, $z$ is the redshift and $\lambda_{emit}$ is the rest-frame wavelength. For a large sample of galaxies this is, however, quite expensive and thus only photometric data are available. For that reason, we will adopt another approach utilising broad band photometric data in looking for so-called "dropouts", a technique first introduced in \cite{Steidel_1996_dropout}. The intrinsic luminosity of dropout galaxies falls off sharply below a specific wavelength and the galaxy will thus be visible in bands with wavelengths longer than the cutoff value, while they will drop out at shorter wavelengths. The cutoff wavelength can either be a Lyman break ($912\,\si{AA}$) or a Balmer break ($3646\,\si{AA}$). As an example, using eq. \ref{redshift}, we would observe the Balmer limit for a galaxy at $z=5$ at $\lambda_{obs}=21876\,\si{AA}\approx2.19\,\si{\mu m}$. Dropouts, with a large break in their continuum flux, have two possible explanations for their red colours: (i) either dust extinction causes strong reddening of otherwise star-forming galaxies, since shorter (bluer) wavelengths are attenuated the most, or (ii) the dropout galaxy is actually a quiescent galaxy that is intrinsically faint below the given wavelength, due to ceased star formation \cite{mo_van_den_bosch_white_2010_MBW_BOOK}. The cut off wavelength for a drop out galaxy from the early universe is longer than that of a low redshift one. A near infrared (NIR)-dropout is therefore not visible in optical or near infrared wavelengths, but can be detected in surveys at wavelengths above $2\,\si{\mu m}$. In this thesis, we therefore look for NIR-dropouts, which in principle will lead to objects with higher redshifts than if we looked for, e.g. optical dropouts. \\
Identifying a dropout galaxy by itself does not guarantee that we have found a galaxy at a high redshift, since it could be either an intrinsically faint galaxy or a galaxy faint due to dust. Nonetheless both findings would be interesting for future research. Finding intrinsically faint galaxies could provide new information about early death (quenching) of massive galaxies, while a dusty galaxy can provide insight into our knowledge of the physical features of more mature galaxies with unprecedented dust contents. Both scenarious will enlighten us on aspects of galaxy evolution that have not been investigated thoroughly so far. \\ \\

The photometric data used for this thesis are from the COSMOS2020 catalogue (Classic and Farmer), which are still under development by a team of astronomers led from NBI (Weaver et. al. \cite{Weaver_2020}). The catalogue is an upgraded version of COSMOS2015 \cite{Laigle_2016}, which incorporates new imaging and spectroscopic data in the COSMOS field. Due to the increase in exposure time, the new catalogue is almost one magnitude deeper in the Visible Infrared
Survey Telescope for Astronomy (VISTA) bands, and includes all IRAC data ever taken in COSMOS. It contains source detection and multi-wavelength photometry for 1.7 million sources across the 2\,$\si{deg}^2$ field. Due to the rareness of the kind of galaxies we are looking for, a large survey is preferable, since we will be able to identify a bigger sample of reliable candidates. On the other hand, we also look for a deep survey due to the faintness of the targets. COSMOS2020, although not a catalogue based on the deepest survey available, has the desired balance between the two requirements, thus providing us with a formidable tool to detect a larger sample of fainter sources. Since we are interested in NIR-dropouts, we will primarily be working with the UltraVISTA bands ($H$ at $1.65\,\si{\mu m}$ and $Ks$ at $2.16\,\si{\mu m}$), along with the IRAC bands: $ch1$ ($\lambda = 3.57\,\si{\mu m}$) and $ch2$ ($\lambda = 4.51\,\si{\mu m}$). \\

This thesis has two separate aims. One is exploring whether a new, better residual map can be produced from the Farmer catalogue. The other main purpose is to devise a new technique for a faster selection of NIR-dropouts. \\
In the article by Weaver et. al. \cite{Weaver_2020}, a new tool for profile-fitting photometric extraction, \textit{The Farmer}, is introduced, which provides additional parameters that we can use for fitting the light profiles of sources, when creating a residual map. The process of developing a residual map of the IRAC $ch1$ image from the 2020 catalogue is described in detail in section 2.1 and the results are reported in section 3.1.\\
Recent work (\cite{Alcalde_Pampliega_2019} and \cite{Wang_2019}) has started a systematic investigation of a few tens of candidate NIR-dropouts. The candidates are found to be mostly massive, dusty galaxies from the early universe, which challenges our understanding of massive galaxy formation. This thesis aim to find such NIR-dropouts among the million of galaxies detected in the COSMOS field. We therefore propose a semi-automated method for the identification of the most reliable candidates of NIR-dropouts. The technique combines unsupervised and semi-supervised machine learning to classify galaxies, utilising in particular t-distibuted Stochastic Neighbour Embedding (t-SNE) for dimensionality reduction and k Nearest Neighbour (kNN) voting for classification. The method for classification of the objects is described in section 2.2.


