\section{Introduction}
Reaching a complete understanding of the formation and evolution of galaxies is one of the central challenges of modern astronomy. To be able to study the early stages of galaxy evolution, we look for progenitors of present-day galaxies. In an astrophysical context, studying the early universe implies looking for distant galaxies with a high redshift, i.e. galaxies that emitted the light we observe at an earlier point in cosmic time. For example a galaxy at redshift $z=5$ emitted light $\sim1$ billion years after the Big Bang\footnote{Using the calculator at \url{http://www.astro.ucla.edu/\%7Ewright/CosmoCalc.html} with $H_0=69.6$, $\Omega_M=0.286$ and $\Omega_{vac}=0.714$. \textcolor{blue}{fix link}}. \\

To identify distant galaxies \textbf{ one can take their spectrum and directly measure the redshift... but this is expensive... for large galaxy samples only photometric data are available. In that case...} look for what is often called "dropout" galaxies. The intrinsic luminosity of these objects falls off sharply below a specific wavelength and the galaxy will thus be visible in bands with wavelengths longer than the cutoff value, while they will "dropout" at shorter wavelengths\footnote{ \textcolor{blue}{Find a better reference but these say something about it:} \url{https://en.wikipedia.org/wiki/Dropout_(astronomy)} and \url{https://ned.ipac.caltech.edu/level5/Illingworth/Ill4.html}}.
A NIR-dropout is therefore not visible in optical or near infra-red wavelengths, but can be detected in surveys at wavelengths above $2 \si{\mu m}$. Galaxies with this feature have two possible explanations for their red colours: (i) either dust extinction causes strong reddening of otherwise star-forming galaxies, since shorter (bluer) wavelengths are attenuated the most, or (ii) the drop out galaxy is actually a quiescent galaxy that is intrinsically faint below the given wavelength, due to ceased star formation \footnote{\textcolor{blue}{How do I refer to a specific page of a book?} p. 79 MBW book }. The cut off wavelength for a drop out galaxy from the early universe is longer than that of a low redshift one, due to the cosmological redshift formula:
\begin{equation}
    \lambda_\mathrm{obs} = (1+z)  \lambda_\mathrm{emit}
\end{equation}
where $\lambda_{obs}$ is the observed wavelength, $z$ is the redshift and $\lambda_{emit}$ is the wavelength the light was emitted at\footnote{Reference? \url{https://en.wikipedia.org/wiki/Redshift#Redshift_formulae} }. We therefore look for NIR-dropouts which likely will result in objects with higher redshifts than if we looked for optical dropouts.\\

Identifying a drop out galaxy by itself doesn't guarantee that we have found a galaxy at a high redshift, since it could be either an intrinsically faint galaxy or a galaxy faint due to external factors. Nonetheless both findings would be interesting for future research. Either it can enlighten us on the early stages of galaxy evolution or it can provide insight to our knowledge of the physical features of more mature galaxies with unprecedented dust contents. To confirm which types of galaxies the drop out candidates, found with this method, belong to, spectroscopic follow up is needed due to the ambiguity of the observations. \\

Recent work (\cite{Alcalde_Pampliega_2019} and \cite{Wang_2019}) has identified such overlooked NIR-dropouts finding them to be massive, dusty galaxies from the early universe. Both quiescent and star-forming galaxies are reported, which challenges our understanding of massive galaxy formation. This thesis proposes a semi-automated method for the identification of the most reliable candidates of NIR-dropout galaxies, thus making the identification process faster. The technique combines unsupervised and semi-supervised machine learning to classify galaxies, utilising in particular t-distibuted Stochastic Neighbour Embedding (t-SNE) for dimensionality reduction and k Nearest Neighbour (kNN) voting for classification. The method for classification of the objects is described in detail in section 2.2 \textcolor{blue}{be sure to refer to the right section}. There are three prerequisites for the technique to produce meaningful results: (i) A large photometric survey containing the coordinates of identified sources \textcolor{red}{do we actually need more? I am not sure what exactly is needed to produce the old residual map. Maybe a radius? To make point source models}, (ii) a residual map created with information from the catalogue and (iii) results from a source detecting algorithm performed on the residual map along with labels on a group of promising candidates. \\

The survey used for this thesis is the latest public COSMOS2015 catalogue containing more than half a million objects over the 2$\si{deg}^2$ COSMOS\footnote{The Cosmic Evolution Survey} field \cite{Laigle_2016}. Due to the rareness of the kind of galaxies we are looking for, a large survey is preferable, since we will be able to identify a bigger sample of reliable candidates. On the other hand, we also look for a deep survey due to the faintness of the targets. COSMOS2015, although not the deepest survey available, has the desired balance between the two measures. Since we are interested in NIR-dropouts, we will primarily be working with the UltraVISTA\footnote{Ultra Deep, near infra red survey with the Visible and Infrared Survey Telescope for Astronomy} bands ($H$ at $1.65\,\si{\mu m}$ and $Ks$ at $2.16\,\si{\mu m}$), along with the IRAC \footnote{\textit{Spitzer}'s Infrared Array Camera} bands: $ch1$ ($\lambda = 3.57\,\si{\mu m}$) and $ch2$ ($\lambda = 4.51\,\si{\mu m}$). \\
The COSMOS2020 catalogue, still under development by a team of astronomers led from NBI (Weaver et. al. \cite{Weaver_2020} \textcolor{red}{is this how I should refer to in house papers? \textbf{Iary modified the sentence}}, is an upgraded version of COSMOS2015 based on new imaging and spectroscopic data in the COSMOS field. The new catalogue is first and foremost almost one magnitude deeper, thus providing an improved ability to detect fainter sources. Furthermore it is larger, containing source detection and multi-wavelength photometry for 1.7 million sources across the 2\,$\si{deg}^2$ field. A new tool for profile-fitting photometric extraction, \textit{The Farmer}, is introduced, which provides additional parameters we can use for fitting the light profiles of the sources, when creating a residual map. The process of developing a residual map from the 2020 catalogue is described in section 2.1 \textcolor{blue}{remember the right subsection here}.\\ 
In this thesis both catalogues are taken advantage of. The COSMOS2020 catalogue will be used for creating a residual map, and the COSMOS2015 catalogue will be used for the classification part. \\
\textcolor{red}{I should also mention something about the residual map of the IRAC image prepared by monetti et al. we ended up using - but do I have any reference for this? Iary said in an email it was prepared with IRACCLEAN, but I can't find any information about this software.}\\

The technique we use for classification of the most reliable candidates of NIR drop out galaxies also requires labels on good objects (and preferably also bad objects), so we can use semi-supervised learning for the clustering. We rely on the previous work studying H-dropout galaxies, and look for objects similar to those in figure 1 in respectively \cite{Alcalde_Pampliega_2019} and \cite{Wang_2019}. \textcolor{red}{Can I refer to the figures in their paper like this? There is an example of how we expect it to look in the results see fig. \ref{embedding_regions}, but being pedantic I don't actually obtain this before the results.} The labeling and use of these seemingly "good candidates" is elaborated in section 2.2 \textcolor{blue}{refer to right section}.

\textcolor{red}{Should I add a last paragraph more structurally going through what will happen in each section of the thesis or are the references to sections along the way enough?}
\textcolor{red}{Perhaps I should make the two parts of the project more clear (1. modelling from Farmer and producing residual map, 2. making classifications with ML). If that is the case I could move the comment saying when whcih catalogue is used to here.}